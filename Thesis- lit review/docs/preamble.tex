% This file contains your LaTeX preamble. A preamble is a part of your document where all required packages and macros can be defined. This needs to be done before the \begin{document} command.

% Documentclass:
% Standard LaTeX classes are: article, book, report, slides, and letter. These cover the basis, but are not best. More advanced users might want to try out the KOMA classes or the memoir class. Optional arguments: 10pt. The font size of the main content is set to 10pt with the option between [].
\documentclass[11pt]{book}
\let\cleardoublepage\clearpage
% Geometry:
% The papersize of the document is defined with the geometry package. Here, the size is set to A4 with a4paper. Other possibilities are a5paper, b5paper, letterpaper, legalpaper and executivepaper.
\usepackage[a4paper]{geometry}
\newcommand\tab[1][1cm]{\hspace*{#1}}
%removes chapter name from the chapters
\renewcommand{\chaptername}{} 

% AMS math packages:
% Required for proper math display.
\usepackage{amsmath,amsfonts,amsthm}

% Graphicx:
% If you want to include graphics in your document, the graphicx package is required.
\usepackage{graphicx}
\graphicspath{ {figs/} }

% Booktabs:
% The booktabs package is needed for better looking tables. 
\usepackage{booktabs}

% SIunitx:
% The SIunitx package enables the \SI{}{} command. It provides an easy way of working with (SI) units.
\usepackage{siunitx}

% URL:
% Clickable URL's can be made with this package: \url{}.
\usepackage{url}

% Caption:
% For better looking captions. See caption documentation on how to change the format of the captions.
\usepackage{caption}

% Hyperref:
% This package makes all references within your document clickable. By default, these references will become boxed and colored. This is turned back to normal with the \hypersetup command below.
\usepackage{hyperref}
	\hypersetup{colorlinks=false,pdfborder=0 0 0}

% Cleveref:
% This package automatically detects the type of reference (equation, table, etc.) when the \cref{} command is used. It then adds a word in front of the reference, i.e. Fig. in front of a reference to a figure. With the \crefname{}{}{} command, these words may be changed.
\usepackage{cleveref}
	\crefname{equation}{equation}{equations}
	\crefname{figure}{figure}{figures}	
	\crefname{table}{table}{tables}
	


\usepackage{float} 

\usepackage{url} 
\newfloat{fig}{thp}{lof}[chapter]
\floatname{fig}{Figure}


